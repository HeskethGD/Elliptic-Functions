\documentclass[]{article}

%opening
\title{Dixon functions are doubly periodic special cases of Kronecker theta functions}
\author{}

\begin{document}

\maketitle

\begin{abstract}
Dixon functions are doubly periodic elliptic functions whereas Kronecker theta functions are ratios of Jacobi theta functions, or equivalently, Weierstrass Sigma functions, that are in general quasi periodic. It is shown that for special cases in which Kronecker theta functions are doubly periodic, they are equivalent to Dixon elliptic functions.
\end{abstract}

\section{}
Dixon functions are a type of elliptic function introduced by Alfred Cardew Dixon. These functions have applications in various areas of mathematics and physics, particularly in the study of elliptic curves, modular forms, and number theory. As elliptic functions, Dixon functions are doubly periodic. Kronecker theta functions are ratios of Jacobi theta functions, or equivalently Weierstrass sigma functions, that are in general quasi periodic. Kronecker theta functions arise in number theory applications as well as solutions to coupled ordinary differential equations that describe physical dynamics such as nonlinear optics, quantum mechanics and general relativity. 
\end{document}
